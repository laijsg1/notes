\documentclass[UTF8]{article}
\usepackage{ctex}
\title{\textbf{编译原理}}
\author{赖钧生}
\begin{document}
\maketitle

\section*{5.1}
\subsection{回顾}
\begin{itemize}
	\item DFA,NFA,正则表达式三者表述能力等价
\end{itemize}


\subsection{从正则表达式直接构造DFA}
\begin{itemize}
	\item nullable:可以空的节点
	\item firstpos:以结点n为根节点为根的子树集合
\end{itemize}
	

\subsection{Lex:词法分析器}
\begin{itemize}
	\item title
	\item 匹配冲突
	\begin{enumerate}
		\item 最长匹配原则
		\item 规定接受状态的优先顺序(?)
	\end{enumerate}
\end{itemize}

\subsection{正规文法到NFA}
\begin{itemize}
	\item 正规文法:
	\begin{enumerate}
		\item 右线性
		\item 左线性
	\end{enumerate}
\end{itemize}


\subsection{NFA到正规文法}
\subsection{上下文无关文法}
\begin{itemize}
	\item $ S\Rightarrow^*\alpha $
	
	$\alpha\in T^*  $
	
	$\alpha\in L(G) $
	
	$\alpha$叫做L(G)的句子
	\item $ S\Rightarrow^*\beta $
	
	$\beta\in (V\cup T)^*  $
	
	$\beta\in L(G) $
	
	$\beta$叫做L(G)的句型
	\item 分析树
\end{itemize}

\subsection{上下文有关文法}

\end{document}
