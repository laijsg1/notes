\documentclass[UTF8]{article}
\usepackage{ctex}
\usepackage{graphicx}

\title{\textbf{编译原理}}
\begin{document}
	\maketitle
	\section{语法分析}
		\subsection{自顶向下分析}
		\begin{enumerate}
			\item 自顶向下分析\\
			从分析树的顶部(根节点)向底部(叶节点)构造分析树\\
			可以看成是从文法开始\textbf{符号S}推导出词串w是过程	\\
			\includegraphics[width=400pt]{picture/3-1-7.png}
			
			
			\item 最左推导\\
			\includegraphics[width=400pt]{picture/3-1-1.png}
			
			\item 最右推导\\
			\includegraphics[width=400pt]{picture/3-1-2.png}
		
			\item 最左推导和最右推导的唯一性\\
			由于他们的每一步所用到的非终结符都是唯一的,所以他们推导出来的也是唯一的。\\
			\includegraphics[width=400pt]{picture/3-1-3.png}
			
			\item 自顶向下的语法分析采用最左推导方式\\
			由于编译器分析串时是自左向右的。\\
			\includegraphics[width=400pt]{picture/3-1-4.png}
			
			\item 自顶向下语法分析的通用形式\\
			\includegraphics[width=400pt]{picture/3-1-5.png}
			
			\item 预测分析\\
			\includegraphics[width=400pt]{picture/3-1-6.png}
		\end{enumerate}
		
		
		\subsection{文法转换}
		\begin{enumerate}
			\item 出现的问题\\
			\includegraphics[width=400pt]{picture/3-2-0.png}\\
			\includegraphics[width=400pt]{picture/3-2-1.png}
			
			\item 消除直接左递归\\
			\includegraphics[width=400pt]{picture/3-2-2.png}\\
			\includegraphics[width=400pt]{picture/3-2-3.png}
			
			
			\item 消除间接左递归\\
			\includegraphics[width=400pt]{picture/3-2-4.png}
			\includegraphics[width=400pt]{picture/3-2-5.png}
			\item 提取左公因子\\
			\includegraphics[width=400pt]{picture/3-2-6.png}
			\includegraphics[width=400pt]{picture/3-2-7.png}
			
			\item 
		\end{enumerate}
		
			
	
\end{document}
