\documentclass[UTF8]{article}
\usepackage{ctex}
\usepackage{graphicx}
\title{数据库第七周课堂笔记}
\author{laisg}
\begin{document}
  \maketitle
  \section{Delete}%
  \label{sec:delete}

  \includegraphics[width=0.8\linewidth]{} 

  \part{连接}%
  \label{prt:连接}
  \paragraph{连接是指把两个表做笛卡尔乘积的基础上加以某些限制}%
  \label{par:连接是指把两个表做笛卡尔乘积的基础上加以某些限制}

  \section{外连接}%
  \label{sec:外连接}
  
  \subsection{左外连接}%
  \label{sub:左外连接}
  
  \subsection{右外连接}%
  \label{sub:外连接}
  
  \subsection{全外连接(和自然连接的区别)}%
  \label{sub:连接}

  \part{Views}%
  \label{prt:views}
  \paragraph{控制一个虚表让不同的用户看到不同的信息}%
  \label{par:_控制一个虚表让不同的用户看到不同的信息}
  \section{Update of a View}%
  \label{sec:update_of_a_view}
  \begin{itemize}
    \item 更新会对应到其对应的实表
  \end{itemize}
  
  
  \section{Materialized Views}%
  \label{sec:materialized_views}
  \begin{itemize}
    \item 这样会使其与原来的实表失去联系
  \end{itemize}

  
  \part{Transaction}%
  \label{prt:transaction}

  \section{commit work(通常来说,是自动的)}%
  \label{sec:commit_work_通常来说_是自动的_}
  
  \section{rollback work}%
  \label{sec:rollback_work}
  
  \part{完整性约束}%
  \label{prt:完整性约束}
  
  
  
  
  
  
  
  
  
  
  

  
  
\end{document}

