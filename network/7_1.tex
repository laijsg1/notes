\documentclass[UTF8]{article}
\usepackage{ctex}
\title{计网第七周笔记}
\author{laisg}
\begin{document}
 \maketitle 
 \part{网络层}%
 \label{prt:网络层}

 \section{虚电路和数据报网络}%
 \label{sec:虚电路和数据报网络}

 \section{路由器工作原理}%
 \label{sec:路由器工作原理}

 \subsection{Router architecture overview}%
 \label{sub:router_architecture_overview}

 \includegraphics[width=0.8\linewidth]{} 

 \begin{enumerate}
   \item Input port functions (排队)\\
   \includegraphics[width=0.8\linewidth]{} 

   \item Switching fabrics (three types)\\
   \includegraphics[width=0.8\linewidth]{} 

   \item Output port (排队)\\
   \includegraphics[width=0.8\linewidth]{} 

   \item 排队
   \begin{itemize}
     \item buffer=$\frac{RTT\cdot C}{\sqrt{N}}$
   \end{itemize}
   
 \end{enumerate}

 \section{IP}%
 \label{sec:ip}

 \paragraph{The Internet Network layer}%
 \label{par:the_internet_network_layer}
 \begin{itemize}
   \item IP
   \item ICMP
   \item 选路协议
 \end{itemize}

 \subsection{IP报文结构}%
 \begin{enumerate}
   \item 报文首部有$32\cdot 8 bit$
   \item 注意IP和TCP的不同
   \item TCP用的是端口号
   \item IP用的是ip号
 \end{enumerate}
 
 \subsection{数据报分片和组装}%
 \label{sub:数据报分片和组装}
 
 \paragraph{由于不同的链路有不同的MTU,所以需要分片}%
 \label{par:由于不同的链路有不同的mtu}
 \begin{itemize}
   \item 为什么偏移量是片的1/8
   \item 为什么每个片都有原来的首部
 \end{itemize}

 \subsection{IP编址 (32bit)}%
 \label{sub:ip编址}

 \begin{enumerate}
   \item 子网
   \item IP地址如何分类
   \begin{itemize}
     \item A 0 1-126\\
     127 用作测试
     \item B 10 128-191
     \item C 110 192-223
     \item 前三个为单播地址
     \item D
     \item E
   \end{itemize}
   \item 特殊用途的IP
   \begin{itemize}
     \item 全为零的IP作为网络本身标识
   \end{itemize}
   \item 网络掩码
   \begin{itemize}
     \item 子网掩码\\
     通过向主机号借位来实现子网掩码
     \item 子网掩码的使用\\
     通过与IP地址进行\textbf{按位与}运算辨别主机属于哪一个网络
   \end{itemize}
 \end{enumerate}

 \section{questions}%
 \label{sec:questions}
 
 \begin{enumerate}
   \item IP是如何分配的
   \begin{itemize}
     \item 静态
     \item 动态(DHCP)
   \end{itemize}
 \end{enumerate}
 

\end{document}

